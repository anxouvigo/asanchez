% Options for packages loaded elsewhere
\PassOptionsToPackage{unicode}{hyperref}
\PassOptionsToPackage{hyphens}{url}
\PassOptionsToPackage{dvipsnames,svgnames,x11names}{xcolor}
%
\documentclass[
  letterpaper,
  DIV=11,
  numbers=noendperiod]{scrartcl}

\usepackage{amsmath,amssymb}
\usepackage{lmodern}
\usepackage{iftex}
\ifPDFTeX
  \usepackage[T1]{fontenc}
  \usepackage[utf8]{inputenc}
  \usepackage{textcomp} % provide euro and other symbols
\else % if luatex or xetex
  \usepackage{unicode-math}
  \defaultfontfeatures{Scale=MatchLowercase}
  \defaultfontfeatures[\rmfamily]{Ligatures=TeX,Scale=1}
\fi
% Use upquote if available, for straight quotes in verbatim environments
\IfFileExists{upquote.sty}{\usepackage{upquote}}{}
\IfFileExists{microtype.sty}{% use microtype if available
  \usepackage[]{microtype}
  \UseMicrotypeSet[protrusion]{basicmath} % disable protrusion for tt fonts
}{}
\makeatletter
\@ifundefined{KOMAClassName}{% if non-KOMA class
  \IfFileExists{parskip.sty}{%
    \usepackage{parskip}
  }{% else
    \setlength{\parindent}{0pt}
    \setlength{\parskip}{6pt plus 2pt minus 1pt}}
}{% if KOMA class
  \KOMAoptions{parskip=half}}
\makeatother
\usepackage{xcolor}
\setlength{\emergencystretch}{3em} % prevent overfull lines
\setcounter{secnumdepth}{-\maxdimen} % remove section numbering
% Make \paragraph and \subparagraph free-standing
\ifx\paragraph\undefined\else
  \let\oldparagraph\paragraph
  \renewcommand{\paragraph}[1]{\oldparagraph{#1}\mbox{}}
\fi
\ifx\subparagraph\undefined\else
  \let\oldsubparagraph\subparagraph
  \renewcommand{\subparagraph}[1]{\oldsubparagraph{#1}\mbox{}}
\fi


\providecommand{\tightlist}{%
  \setlength{\itemsep}{0pt}\setlength{\parskip}{0pt}}\usepackage{longtable,booktabs,array}
\usepackage{calc} % for calculating minipage widths
% Correct order of tables after \paragraph or \subparagraph
\usepackage{etoolbox}
\makeatletter
\patchcmd\longtable{\par}{\if@noskipsec\mbox{}\fi\par}{}{}
\makeatother
% Allow footnotes in longtable head/foot
\IfFileExists{footnotehyper.sty}{\usepackage{footnotehyper}}{\usepackage{footnote}}
\makesavenoteenv{longtable}
\usepackage{graphicx}
\makeatletter
\def\maxwidth{\ifdim\Gin@nat@width>\linewidth\linewidth\else\Gin@nat@width\fi}
\def\maxheight{\ifdim\Gin@nat@height>\textheight\textheight\else\Gin@nat@height\fi}
\makeatother
% Scale images if necessary, so that they will not overflow the page
% margins by default, and it is still possible to overwrite the defaults
% using explicit options in \includegraphics[width, height, ...]{}
\setkeys{Gin}{width=\maxwidth,height=\maxheight,keepaspectratio}
% Set default figure placement to htbp
\makeatletter
\def\fps@figure{htbp}
\makeatother

\KOMAoption{captions}{tableheading}
\makeatletter
\makeatother
\makeatletter
\makeatother
\makeatletter
\@ifpackageloaded{caption}{}{\usepackage{caption}}
\AtBeginDocument{%
\ifdefined\contentsname
  \renewcommand*\contentsname{Table of contents}
\else
  \newcommand\contentsname{Table of contents}
\fi
\ifdefined\listfigurename
  \renewcommand*\listfigurename{List of Figures}
\else
  \newcommand\listfigurename{List of Figures}
\fi
\ifdefined\listtablename
  \renewcommand*\listtablename{List of Tables}
\else
  \newcommand\listtablename{List of Tables}
\fi
\ifdefined\figurename
  \renewcommand*\figurename{Figure}
\else
  \newcommand\figurename{Figure}
\fi
\ifdefined\tablename
  \renewcommand*\tablename{Table}
\else
  \newcommand\tablename{Table}
\fi
}
\@ifpackageloaded{float}{}{\usepackage{float}}
\floatstyle{ruled}
\@ifundefined{c@chapter}{\newfloat{codelisting}{h}{lop}}{\newfloat{codelisting}{h}{lop}[chapter]}
\floatname{codelisting}{Listing}
\newcommand*\listoflistings{\listof{codelisting}{List of Listings}}
\makeatother
\makeatletter
\@ifpackageloaded{caption}{}{\usepackage{caption}}
\@ifpackageloaded{subcaption}{}{\usepackage{subcaption}}
\makeatother
\makeatletter
\@ifpackageloaded{tcolorbox}{}{\usepackage[many]{tcolorbox}}
\makeatother
\makeatletter
\@ifundefined{shadecolor}{\definecolor{shadecolor}{rgb}{.97, .97, .97}}
\makeatother
\makeatletter
\makeatother
\ifLuaTeX
  \usepackage{selnolig}  % disable illegal ligatures
\fi
\IfFileExists{bookmark.sty}{\usepackage{bookmark}}{\usepackage{hyperref}}
\IfFileExists{xurl.sty}{\usepackage{xurl}}{} % add URL line breaks if available
\urlstyle{same} % disable monospaced font for URLs
\hypersetup{
  colorlinks=true,
  linkcolor={blue},
  filecolor={Maroon},
  citecolor={Blue},
  urlcolor={Blue},
  pdfcreator={LaTeX via pandoc}}

\author{}
\date{}

\begin{document}
\ifdefined\Shaded\renewenvironment{Shaded}{\begin{tcolorbox}[boxrule=0pt, interior hidden, borderline west={3pt}{0pt}{shadecolor}, enhanced, breakable, frame hidden, sharp corners]}{\end{tcolorbox}}\fi

\hypertarget{optimizaciuxf3n-do-custo-de-fabricaciuxf3n-dun-tanque}{%
\section{Optimización do custo de fabricación dun
tanque}\label{optimizaciuxf3n-do-custo-de-fabricaciuxf3n-dun-tanque}}

Os enxeñeiros químicos (así como outros especialistas como enxeñeiros
mecánicos e civís) adoitan enfrontarse ao problema xeral do deseño de
buques que transportan líquidos ou gases.

\hypertarget{enunciado-do-problema}{%
\subsection{Enunciado do problema}\label{enunciado-do-problema}}

Supón que se che pide que determines as dimensións dun pequeno depósito
cilíndrico para o transporte de residuos tóxicos que se transportará nun
camión. O obxectivo xeral é minimizar o custo do tanque. Non obstante,
ademais do custo, debe asegurarse de que pode manter a cantidade de
líquido requirida e que non supere as dimensións do camión. Debido a que
o tanque transportará residuos tóxicos, é necesario que teña un espesor
determinado, dentro de certas normativas.

Na figura 1 se amosa un esquema do tanque e da caixa. Como se pode ver
nela, o tanque é un cilindro con dúas placas soldadas a cada extremo.

Figura 1. Determinación das dimensións óptimas dun tanque cilíndrico
para o transorte de lixo tóxico.

O custo do tanque ten dous compoñentes:

\begin{enumerate}
\def\labelenumi{\arabic{enumi}.}
\tightlist
\item
  Custos de materiais baseados no peso
\item
  Custos de soldadura baseados na lonxitude a soldar. Fíxate que este
  último consiste en soldar tanto a xunta interior como a xunta externa
  onde as placas se atopan co cilindro.
\end{enumerate}

Os datos necesarios para resolver o problema resúmense na táboa 1.

\begin{longtable}[]{@{}llrl@{}}
\toprule()
Parámetro & Símbolo & Valor & Unidades \\
\midrule()
\endhead
Volume requirido & V0 & 0.8 & m3 \\
Espesor & t & 3.0 & cm \\
Densidade & \(\rho\) & 8000.0 & kg/m3 \\
Lonxitude de la caixa & Lmáx & 2.0 & m \\
Ancho de la caixa & Dmáx & 1.0 & m \\
Custo do material & Cm & 4.5 & €/kg \\
Custo de soldadura & Cw & 20.0 & €/m \\
\bottomrule()
\end{longtable}

\hypertarget{soluciuxf3n}{%
\subsection{Solución}\label{soluciuxf3n}}

O obxectivo aquí é construír un tanque a un custo mínimo. O custo está
relacionado coas variables de deseño (lonxitude e diámetro), xa que
teñen un efecto sobre a masa do tanque e as lonxitudes a soldar.
Ademais, o problema ten restricións, xa que o tanque debe:

\begin{enumerate}
\def\labelenumi{\arabic{enumi}.}
\tightlist
\item
  caber no camión.
\item
  ter capacidade para o volume de material requirido.
\end{enumerate}

O custo derívase dos custos do material do tanque e da soldadura. Polo
tanto, a función obxectiva está formulada como unha minimización: \[
C = c_m+c_w l_w
\] onde \(C\) = custo (€), \(m\) = masa (kg), \(l_w\) = lonxitude a
soldar (m), \(c_m\) e \(c_w\) son factores de custo por masa (€\$/kg) e
lonxitude de soldadura (\hspace{0pt}€/m), respectivamente.

Despois, se relacionan a masa e a lonxitude de soldadura coas dimensións
do tambor. Primeiro, calcúlase a masa como o volume do material pola súa
densidade. O volume de material usado para construír as paredes laterais
(é dicir, o cilindro) calcúlase do seguinte xeito: \[
\begin{align*}
V_{cilindro} = L \pi \left [ {\left ( \frac {D}{2} + t \right )}^2 - {\left ( \frac {D}{2} \right )}^2 \right ]
\end{align*}
\] Para cada placa circular nos extremos: \[
V_{placa} = \pi { \left ( \frac {D}{2} +t  \right )}^2 t
\] Deste xeito, a masa calcúlase segundo: \[
m=\rho\left\{L \pi\left[\left(\frac{D}{2}+t\right)^{2}-\left(\frac{D}{2}\right)^{2}\right]+2 \pi\left(\frac{D}{2}+t\right)^{2} t\right\}
\] onde \(\rho\) é a densidade (kg/m3) .

A lonxitude de soldadura para unir cada placa é igual á circunferencia
interna e externa do cilindro. Para as dúas placas, a lonxitude total da
soldadura será: \[
l_w = 2 \left [ 2 \pi \left ( \frac {D}{2} + t \right ) + 2 \pi \frac {D}{2} \right ] = 4 \pi (D + t)
\] Dados os valores de \(D\) e \(L\) (lembra que o grosor \(t\) está
establecido por regulación), as ecuacións (1), (2) e (3) proporcionan un
medio para calcular o custo. Ten en conta tamén que cando as ecuacións
(2) e (3) substitúense pola ecuación (1), A función obxectivo que se
obtén non é lineal.

Despois formúlanse as restricións. En primeiro lugar, débese calcular o
volume que pode almacenar o tanque acabado: \[
V = \frac {\pi D^2}{4} L
\] O valor debe ser igual ó volume desexado. Deste xeito, unha
restrición é: \[
\frac {\pi D^2 L}{4} = V_0
\] onde \(V_0\) é o volume desexado (m3).

As restricións restantes teñen que ver con que o tanque se axuste ás
dimensións da caixa do camión: \[
\begin{align*}
L \leq L_{máx} \\
D \leq D_{máx}
\end{align*}
\] O problema agora está especificado. Coa substitución dos valores da
táboa 1, resúmese como:

Maximizar \(C\) = 4.5\(m\) + 20\(l_w\)

suxeito a: \[
\begin{align*}
\frac {\pi D^2 L}{4} & = & 0.8 \\
L & \leq & 2 \\
D & \leq & 1
\end{align*}
\] onde: \[
\begin{align*}
m = 8000 \left\{ \left [ L \pi {\left ( \frac {D}{2} + 0.03 \right)}^2 - {\left ( \frac {D}{2} \right )}^2 \right] + 2 \pi {\left ( \frac {D}{2} + 0.03 \right )}^2 0.03 \right\}
\end{align*}
\] e: \[
\begin{align*}
l_w = 4 \pi (D + 0.03)
\end{align*}
\] Agora o problema pódese resolver de diferentes xeitos. Non obstante,
o método máis sinxelo para un problema desta magnitude é empregar unha
ferramenta como Solver de Excel.



\end{document}
